%%%%%%%%%%%%%%%%%%%%%%%%%%%%
%% IHEID CHAPTER TEMPLATE %%
%%%%%%%%%%%%%%%%%%%%%%%%%%%%

%%%% PAGE LAYOUT
% Pass this in from YAML
\documentclass[a4paper, oneside]{report}
\usepackage[includehead,hmargin={3.1cm, 3.1cm}, vmargin={2.5cm,2.7cm}, headsep=.8cm,footskip=1.2cm]{geometry}

\usepackage{fancyhdr}
\setlength{\headheight}{15pt}
\fancyhf{} % clear the header and footers
\pagestyle{fancy}
\renewcommand{\chaptermark}[1]{\markboth{\thechapter. #1}{\thechapter. #1}}
\renewcommand{\sectionmark}[1]{\markright{\thesection. #1}} 
\renewcommand{\headrulewidth}{0pt}
\fancyhead[LO]{\emph{\leftmark}}

\fancypagestyle{plain}{\fancyhf{}\fancyfoot[C]{\emph{\thepage}}}

% This adds a "DRAFT" footer to every normal page.  (The first page of each chapter is not a "normal" page.)
\fancyfoot[C]{\emph{DRAFT: \today}}
\fancyfoot[R]{\emph{\thepage}}

%%%%% FONTS
\RequirePackage[T1]{fontenc} % requires XeLatex or LuaTex (remove to use pdfLaTex)
%\RequirePackage[utf8]{inputenc} % ignored when using XeLaTex or LuaTex (uncomment when using pdfLaTex)
\RequirePackage{microtype} % this makes fonts almost imperceptibly smoother
\RequirePackage{fontspec} % requires XeLatex or LuaTex (remove to use pdfLaTex)
% For the headings we will use Helvetica
\RequirePackage{helvet}
\usepackage[immediate]{silence}
\WarningFilter[temp]{latex}{Command} % silence the warning
\usepackage{sectsty}
\DeactivateWarningFilters[temp] % So nothing unrelated gets silenced
\allsectionsfont{\sffamily}
% For the main text and equations we will use Baskerville and Palatino
\RequirePackage{mathpazo}
\RequirePackage{baskervald}
% For formatting code or package names, we will use Lucida Console
\RequirePackage{zi4}
% Enable strikethrough
\usepackage[normalem]{ulem}

% This highlights (in blue) corrections marked with (for words) \mccorrect{blah} or (for whole
% paragraphs) \begin{mccorrection} . . . \end{mccorrection}.  This can be useful for sending a PDF of
% your corrected thesis to your examiners for review.  Turn it off, and the blue disappears.
% 
% UL 30 Nov 2018 pandoc puts lists in 'tightlist' command when no space between bullet points in Rmd file
\providecommand{\tightlist}{%
  \setlength{\itemsep}{0pt}\setlength{\parskip}{0pt}}
 
%UL 15 Oct 2019, enable link highlighting to be turned off from YAML
% \usepackage[dvipsnames]{color}
% \usepackage[colorlinks=true,pdfpagelabels,hidelinks=false]{hyperref}
\RequirePackage[colorlinks=true,linkcolor=red,
citecolor=red,filecolor=red,urlcolor=black]{hyperref} % uses IHEID red for external links
% \hypersetup{citecolor=red}

%%%%% BIBLIOGRAPHY SETUP
% Note that your bibliography will require some tweaking depending on your department, preferred format, etc.
% The options included below are just very basic "sciencey" and "humanitiesey" options to get started.
% If you've not used LaTeX before, I recommend reading a little about biblatex/biber and getting started with it.
% If you're already a LaTeX pro and are used to natbib or something, modify as necessary.
% Either way, you'll have to choose and configure an appropriate bibliography format...

% The science-type option: numerical in-text citation with references in order of appearance.
% \usepackage[style=numeric-comp, sorting=none, backend=biber, doi=false, isbn=false]{biblatex}
% \newcommand*{\bibtitle}{References}

% The humanities-type option: author-year in-text citation with an alphabetical works cited.
% \usepackage[style=authoryear, sorting=nyt, backend=biber, maxcitenames=2, useprefix, doi=false, isbn=false]{biblatex}
% \newcommand*{\bibtitle}{Works Cited}

%UL 3 Dec 2018: set this from YAML in index.Rmd
% % \usepackage[style=numeric-comp, sorting=none, backend=biber, doi=, isbn=false]{biblatex}
% \newcommand*{\bibtitle}{References}
% 
% This makes the bibliography left-aligned (not 'justified') and slightly smaller font.
% \renewcommand*{\bibfont}{\raggedright\small}

% Change this to the name of your .bib file (usually exported from a citation manager like Zotero or EndNote).
% \addbibresource{bib/references.bib}

% Uncomment this if you want equation numbers per section (2.3.12), instead of per chapter (2.18):
%\numberwithin{equation}{subsection}

\newlength{\cslhangindent}
\setlength{\cslhangindent}{1.5em}
\newenvironment{cslreferences}%
  {\setlength{\parindent}{0pt}%
  \everypar{\setlength{\hangindent}{\cslhangindent}}\ignorespaces}%
  {\par}
  

%%%%% THESIS / TITLE PAGE INFORMATION
% Everybody needs to complete the following:
\title{Conclusion}
\author{Chin Man KWAN}

% JEM: Lengths for single spacing (ie separate abstract, captions), front matter (abstract,
%   acknowledgements, table of contents, etc), and main body text.
\newlength{\singlebaselineskip}
\newlength{\frontmatterbaselineskip}
\newlength{\textbaselineskip}

\setlength{\singlebaselineskip}{\baselineskip}
\setlength{\frontmatterbaselineskip}{17pt plus1pt minus1pt}
\setlength{\textbaselineskip}{22pt plus2pt}

%
% Environments
%

% Managing minitoc for chapters
% \usepackage[nohints,tight]{minitoc} 
% 	\setcounter{minitocdepth}{2} 
% 	% Generates mini tables of contents per chapter
% 	
% % JEM: The following fixes minitoc to not be multiple-spaced.
% \makeatletter
% \let\oldmtc@verse\mtc@verse
% \renewcommand{\mtc@verse}[1]{\oldmtc@verse{#1}\setlength{\baselineskip}{\z@}}
% \makeatother
	
% For the abbreviations
\usepackage{enumitem}
\newlist{abbrv}{itemize}{1}
\setlist[abbrv,1]{label=,labelwidth=1in,align=parleft,itemsep=0.1\baselineskip,leftmargin=!}

\usepackage{nicefrac}
\usepackage{amsmath}
\usepackage{amssymb}
\usepackage{textcomp}

\usepackage{longtable} 
	% Allows tables to span multiple pages (this package must be called before hyperref)

\usepackage[font=small,labelfont=bf]{caption} 
	% Nicer captions

\usepackage{multicol,multirow,array} 
	% Used to make multiple columns for the indices and for creating columns that span multiple rows in tables

\usepackage{rotating} 
	% To allow tables in landscape mode

\usepackage{booktabs} 
	% For better looking tables

\usepackage{pdfpages} 
	% Allows multi-page pdfs to be inserted as graphics
	
\usepackage{graphicx}
\graphicspath{{../figures/}{_bookdown_files/}}

\usepackage{xfrac}

%%%%%%%%%%%%%%%%%%%%%%%%%%%
%% ACTUAL DOCUMENT HERE  %%
%%%%%%%%%%%%%%%%%%%%%%%%%%%
\begin{document}

\maketitle



\tableofcontents

%%%%% CHOOSE YOUR LINE SPACING HERE
% This is the official option.  Use it for your submission copy and library copy:
\setlength{\textbaselineskip}{22pt plus2pt}
% This is closer spacing (about 1.5-spaced) that you might prefer for your personal copies:
%\setlength{\textbaselineskip}{18pt plus2pt minus1pt}

% You can set the spacing here for the roman-numbered pages (acknowledgements, table of contents, etc.)
\setlength{\frontmatterbaselineskip}{17pt plus1pt minus1pt}

% UL: You can set the general paragraph spacing here - I've set it to 2pt (was 0) so
% it's less claustrophobic
\setlength{\parskip}{2pt plus 1pt}

% Leave this line alone; it gets things started for the real document.
\setlength{\baselineskip}{\textbaselineskip}

%%%%% CHAPTERS
% Add or remove any chapters you'd like here, by file name (excluding '.tex'):
\flushbottom

% all your chapters and appendices will appear here
\hypertarget{examiner-aragorn-ii-elessar-only-required-if-phd-otherwise-delete-or-comment-out}{%
\chapter{examiner: Aragorn II Elessar \# only required if PhD; otherwise
delete or comment
out}\label{examiner-aragorn-ii-elessar-only-required-if-phd-otherwise-delete-or-comment-out}}

Placeholder

\hypertarget{metadata}{%
\section{Metadata}\label{metadata}}

\hypertarget{front-matter}{%
\section{Front Matter}\label{front-matter}}

\hypertarget{bibliography}{%
\section{Bibliography}\label{bibliography}}

\hypertarget{pdf-formatting}{%
\section{PDF Formatting}\label{pdf-formatting}}

\hypertarget{output-options}{%
\section{Output Options}\label{output-options}}

\hypertarget{intro}{%
\chapter{Introduction}\label{intro}}

Placeholder

\hypertarget{theory}{%
\chapter{Literature Review}\label{theory}}

Placeholder

\hypertarget{theories}{%
\section{Theories}\label{theories}}

\hypertarget{what-is-national-identity}{%
\subsection{What is national
identity?}\label{what-is-national-identity}}

\hypertarget{impact-of-conceptions-of-national-identity-on-attitudes-towards-immigrants}{%
\subsection{Impact of conceptions of national identity on attitudes
towards
immigrants}\label{impact-of-conceptions-of-national-identity-on-attitudes-towards-immigrants}}

\hypertarget{chineseness-in-hong-kong}{%
\subsection{Chineseness in Hong Kong}\label{chineseness-in-hong-kong}}

\hypertarget{empirics}{%
\section{Empirics}\label{empirics}}

\hypertarget{how-do-different-political-camps-in-hong-kong-perceive-national-identity}{%
\subsection{How do different political camps in Hong Kong perceive
national
identity?}\label{how-do-different-political-camps-in-hong-kong-perceive-national-identity}}

\hypertarget{the-pro-beijing-camp}{%
\subsubsection{The pro-Beijing camp}\label{the-pro-beijing-camp}}

\hypertarget{the-pro-democracy-camp-or-pan-democrats}{%
\subsubsection{The pro-Democracy camp (or
pan-democrats)}\label{the-pro-democracy-camp-or-pan-democrats}}

\hypertarget{the-localist-camp}{%
\subsubsection{The localist camp}\label{the-localist-camp}}

\hypertarget{asylum-seekers-in-hong-kong}{%
\subsection{Asylum seekers in Hong
Kong}\label{asylum-seekers-in-hong-kong}}

\hypertarget{hong-kongs-policy-regime}{%
\subsubsection{Hong Kong's policy
regime}\label{hong-kongs-policy-regime}}

\hypertarget{responses-by-political-camps}{%
\subsubsection{Responses by political
camps}\label{responses-by-political-camps}}

\setcounter{chapter}{2}

\hypertarget{methods}{%
\chapter{Methods}\label{methods}}

After reviewing both the theories of the impact of national identity on
xenophobia as well as how political camps in Hong Kong imagine the
city's national identity, how could we know the possible associations
between each camp's definition of the city's national identity and its
attitude towards asylum seekers who are mainly non-ethnic Chinese? This
section will discuss the strategy of answering this question.

\hypertarget{hypotheses}{%
\section{Hypotheses}\label{hypotheses}}

Based on the aforementioned literature in the previous section, I will
formulate hypotheses to be tested and explain why I expect them to go in
certain directions. The main hypothesis to be tested will be:

\begin{quote}
H\textsubscript{1}: Pro-Beijing media outlets are more likely to
negatively portray asylum seekers than media outlets from other
political camps.
\end{quote}

I expect the pro-Beijing camp news media to frame asylum seekers more
negatively based on the literature on the relationship between national
identity and xenophobia (e.g. Hjerm
(\protect\hyperlink{ref-hjermNationalIdentityComparison1998}{1998}) and
Heath and Tilley
(\protect\hyperlink{ref-heathBritishNationalIdentity2005}{2005}) ) which
states that a more ethnic-based conception of national identity is
generally associated with a higher degree of xenophobia towards
immigrants, of which asylum seekers is one of the (most vulnerable)
subgroups of immigrants. Accordingly, since pro-Beijing politicians and
officials (including the Chinese President \emph{Xi Jinping} himself)
(\protect\hyperlink{ref-linMakeChinaGreat2021}{Lin and Jackson 2021})
resort heavily to ethnic-based discourses for justifying Chineseness of
the Hong Kong identity, pro-Beijing media outlets can be expected to
hold more negative attitudes towards the mostly non-ethnic Chinese
non-refoulement claimants. In particular, the pro-Beijing camp's
emphasis on how Hong Kong citizens belong to the greater Chinese
national identity based on shared ethnicity with mainland Chinese might
lead to its affiliated media outlets to portray asylum seekers as a
threat to society since their ethnic distinctiveness marks them as
ever-being strangers of the family of the majority Chinese population as
Erni (\protect\hyperlink{ref-erniWhoNeedsStrangers2012}{2012})
describes.

By contrast, since the pro-democracy camp focuses more on civic values
(e.g.~civil liberties, the rule of law) while defining the Hong Kong
identity, the non-Chineseness of asylum seekers should appear less
threatening to the Hong Kong identity from the pro-democracy camp's
point of view. Lastly, the localist camp's primary focus on the
mainland-Hong-Kong division regarding the city's identity again prevents
a clear direction for H\textsubscript{1} from being formulated
beforehand.

\hypertarget{data-collection}{%
\section{Data collection}\label{data-collection}}

I will study news media reports on asylum seekers and non-refoulement
claimants in Hong Kong by local mass media during the year of 2019.
Specifically, media publishing in \textbf{Chinese} in printed format
will be included in the data\footnote{The full list of the included
  newspaper outlets is provided in \emph{table 3.1}.}. For the retrieval
of the relevant newspaper articles, I will rely on the \emph{WiseNews}
database which can fetch news articles with search keywords published in
Hong Kong\footnote{I would like to thank my friend in Hong Kong helping
  me gain access to the database, which would otherwise have been
  extremely difficult.}. While it would have been more ideal to also
include online news media into the dataset as Lee
(\protect\hyperlink{ref-leeChangingPoliticalEconomy2018}{2018}) mentions
that they have become quite prominent in the 2010s for providing
counter-narratives to the mainstream media which are often co-opted by
the Chinese Government\footnote{It should be noted, however, that by the
  very beginning of 2022, several online media outlets such as
  \emph{Stand News} has defuncted after the flagship pro-Democracy
  newspaper outlet \emph{Apple Daily} had dissolved in June 2021. The
  implication of this phenomenon to the media's narrative on asylum
  seekers in Hong Kong will be discussed further in the conclusion
  section.}, WiseNews does not directly provide the texts of online
media news article and thus requires additional steps of web scrapping
which are, unfortunately, beyond the author's capacity by the time of
writing. That being said, including only the printed news media during
2019 would still cover 16 of the well-known news media with associations
to different political camps according to the survey on media
credibility by the Chinese University of Hong Kong
(\protect\hyperlink{ref-cuhkPublicEvaluationMedia2020}{CUHK 2020}). In
other words, just searching for printed news media outlets still allows
me to discover whether the attitudes of reporting on asylum seekers by
some of the most well-known and widely circulated media outlets in Hong
Kong might be related to their political associations.

The reason for choosing to collect media reports published in 2019 is
because it was when the Anti-Extradition Law Amendment protest first
broke out in June and then lasted for several months, a period when one
can argue the manifestation of political ideologies should be the most
salient for all the political camps. This should indeed be the case for
both the pro-democracy camp whose supporters participated in several
large-scale protests and sometimes even engaged in numerous violent
clashes with the Hong Kong police to oppose a law that would have
permitted extradition of fugitives to mainland China, and the
pro-Beijing camp whose authority was once again immensely challenged
after the 2014 Umbrella Movement. Accordingly, it would be interesting
to investigate whether the issue of asylum seekers still remained its
salience and was reported in similar manners given the co-occurrence of
another arguably more salient political event (i.e.~the anti-Extradition
Law protests), as compared to when the issue first had become prominent
back in 2016. Specifically, analysing newspaper articles in 2019
provides an opportunity to observe whether and how asylum seekers in
Hong Kong might be represented by the media in tandem with major
political events.

It is worth discussing some of the limitations of using media reports as
the data for this thesis. For starters, media reports do not necessarily
reflect how individuals identifying with each political camp might
perceive asylum seekers directly and might not be fit for investigating
the causal mechanism of how definition of Hong Kong's national identity
may cause a political camp to frame asylum seekers in particular ways.
Despite these shortcomings, media reports can still adequately help
prompt the hypothesis of whether depictions of asylum seekers in Hong
Kong might be associated with their political camp's definition of the
city's national identity. In fact, previous research about the framing
of asylum seekers by mass media
(\protect\hyperlink{ref-cooperMediaCoverageRefugees2017}{Cooper et al.
2017}; \protect\hyperlink{ref-mckayAnyOneThese2011}{McKay, Thomas, and
Warwick Blood 2011}; \protect\hyperlink{ref-ngFramingIssueAsylum2019}{I.
Ng, Choi, and Chan 2019};
\protect\hyperlink{ref-odohertyAsylumSeekersBoat2007}{O'Doherty and
Lecouteur 2007}) point to its potential impact on affecting how the
public may perceive the issue via being a major provider of related
information, and sometimes these media reports may even serve to justify
policymaking or reflect the government's position on the issue. Drawing
from Fong
(\protect\hyperlink{ref-fongInbetweenLiberalAuthoritarianism2017}{2017})
and Lee (\protect\hyperlink{ref-leeChangingPoliticalEconomy2018}{2018}),
the media industry in Hong Kong is quite remarked for their political
stances, and quite a number of outlets have their owners being
affiliated with the Chinese and/or HKSAR Governments. Therefore, reports
published by media outlets should be able to reflect political camps'
attitudes towards asylum seekers and, perhaps more importantly, the
version of framing that they would like to persuade their readers to
accept, thereby making them a suitable data source to test the
hypotheses proposed above.

Language-wise, including only Chinese-language news media omits how
asylum seekers may be represented by English-language media in Hong Kong
(e.g.~the South China Morning Post) and thus cannot investigate whether
the representation of asylum seekers by media outlets to potential
non-Chinese audience may also vary based on each outlet's political
affiliation. Moreover, solely focusing on textual content of the media
reports means discarding graphical and (if also published online)
audio-visual materials accompanied in a given news report, which may
contain additional information that are not observed in the texts of
news articles. Nevertheless, analysing the textual content of the media
reports should still yield a considerable amount of information about
the representation of asylum seekers in Hong Kong by media outlets
linked to different political camps, since after all a large proportion
of the content in news reports are represented textually.

To search for relevant media reports on WiseNews, I will refer to the
keywords used by I. Ng, Choi, and Chan
(\protect\hyperlink{ref-ngFramingIssueAsylum2019}{2019}) which refer to
the group of population-in-interest. These terms can either be directly
describing this group of population (e.g.~refugees, illegal immigrants)
or policies related to asylum seeking (e.g.~non-refoulement claim,
immigration paper). \emph{Table 3.1} lists all the Chinese terms used
for searching the relevant news articles on WiseNews and their meanings
in English. Additional notes are used for indicating whether some
articles were omitted and why. The author also omitted articles only
mentioning the keywords briefly throughout the main text which discusses
issues not directly related to asylum seekers. As for the keywords
``South Asian'', ``illegal immigration'' and ``human snake'', the author
only included news entries fetched from these keywords if at least one
of the other keywords directly referring to asylum seekers
(i.e.~non-refoulement claim(ant), torture claim(ant), asylum seeking,
refugee and immigration paper) co-appeared in either the title or main
text. This is to ensure that the news articles in the dataset are
obviously describing non-refoulement claimants in Hong Kong.

\begin{longtable}[]{@{}
  >{\raggedright\arraybackslash}p{(\columnwidth - 4\tabcolsep) * \real{0.1809}}
  >{\raggedright\arraybackslash}p{(\columnwidth - 4\tabcolsep) * \real{0.3404}}
  >{\raggedright\arraybackslash}p{(\columnwidth - 4\tabcolsep) * \real{0.4787}}@{}}
\caption{Translation of the keywords used to search for media reports on
WiseNews}\tabularnewline
\toprule
\begin{minipage}[b]{\linewidth}\raggedright
Original terms in Chinese pinyin
\end{minipage} & \begin{minipage}[b]{\linewidth}\raggedright
Literal meaning in English
\end{minipage} & \begin{minipage}[b]{\linewidth}\raggedright
Notes
\end{minipage} \\
\midrule
\endfirsthead
\toprule
\begin{minipage}[b]{\linewidth}\raggedright
Original terms in Chinese pinyin
\end{minipage} & \begin{minipage}[b]{\linewidth}\raggedright
Literal meaning in English
\end{minipage} & \begin{minipage}[b]{\linewidth}\raggedright
Notes
\end{minipage} \\
\midrule
\endhead
mian3 qian3fan3 sheng1qing3 & Non-refoulement claim(ant) & \\
ku4xing2 sheng1qing3 & Torture claim(ant) & \\
fei1fa3 ru4jing4, ren2she2 & Illegal immigration, human snake (alias of
illegal immigrants) & Excluded illegal immigration from mainland
China \\
xun2qiu2 bi4hu4 & Asylum seeking & Excluded asylum seekers \emph{outside
of} Hong Kong \\
nan4min2 & Refugee & Excluded refugees \emph{outside of} Hong Kong and
Hong Kong activists seeking refuge overseas \\
nan2ya4 & South Asian & Articles about the South Asian region are
excluded. \\
xing2 jie1 zhi3 & Immigration paper & Documents issued by the
Immigration Department to non-refoulement claimants. \\
\bottomrule
\end{longtable}

\hypertarget{modelling}{%
\section{Modelling}\label{modelling}}

Since the period under study spans for the year 2019, the dataset
contains almost 700 articles in total which would be extremely time- and
resource-consuming to qualitatively analyse each article line by line,
whereas only focusing on a small subset of the articles may risk losing
generalisability. On the other hand, while it is true that using human
coding (whether it be done by trained experts or crowd coding platforms)
for sentiment analysis will be likely to yield the most accurate results
as van Atteveldt, van der Velden, and Boukes
(\protect\hyperlink{ref-vanatteveldtValiditySentimentAnalysis2021}{2021})
have noted, doing so will be too resource-intensive for a master-level
thesis, and this method may also fall short of reliability, i.e.~how
other researchers may replicate the study if wanted. Despite some of the
caveats listed by van Atteveldt, van der Velden, and Boukes
(\protect\hyperlink{ref-vanatteveldtValiditySentimentAnalysis2021}{2021})
(e.g.~accidentally identifying spurious relationships between variables,
problem of overfitting) which can be remedied with corresponding
solutions, I will use supervised machine learning to test the
hypothesis. This means H\textsubscript{1} will be treated as a sentiment
analysis problem which classifies the \emph{polarity} of the news
articles (i.e.~whether they reported on asylum seekers \emph{positively,
neutrally or negatively}). How the independent (both the media outlet's
political camp and other controls) and dependent variables are
operationalised and coded will be discussed in the next section, and for
now I will talk about what kinds of tools and models will be used for
the analysis.

Starting with the models that will be used for the analysis, I will rely
on some commonly used classification models. Candidate models include
logistic regression, support vector machine (SVM), random forest and
gradient boosting. Although van Atteveldt, van der Velden, and Boukes
(\protect\hyperlink{ref-vanatteveldtValiditySentimentAnalysis2021}{2021})
also mention that deep learning is another popular choice of machine
learning model for sentiment analysis tasks, I did not use such a model
for this paper due to the relatively small size of the dataset at hand
as well as the difficulty of interpreting the features' impact on the
dependent variable directly from neural network models. I will first run
some baseline models with the default parameter settings of the models,
and then select one for further tuning based on their performance on
their \emph{f1 scores} (which measures the harmonic mean of
\emph{precision} and \emph{recall})\footnote{Just a reminder, the
  formula of calculating a model's F1 score is:
  \(2\cdot\frac{\left(precision\cdot recall\right)}{\left(precision+recall\right)}\)
  (\protect\hyperlink{ref-pedregosaScikitlearnMachineLearning2011}{Pedregosa
  et al. 2011}).}. Using f1 score as the evaluation metric can better
measure a model's performance when there is class imbalance in the
dependent variable compared to metrics like accuracy and the ROC-AUC
score.

The model which obtains the overall highest f1 score will then be used
for calculating the \emph{SHAP values} of the features
(\protect\hyperlink{ref-lundbergSlundbergShap2022}{Lundberg 2022}) which
evaluate how much impact each feature has on the model prediction when
the features is at certain values versus at its baseline value. In
essence, the higher the SHAP value of a feature, the higher its impact
of the model's prediction.

In order to pre-process the articles into suitable formats as machine
learning model inputs, I will also transform the news articles into a
term-document matrix which records the frequency of each word
(i.e.~token) appearing in each article (i.e.~document). Instead of using
a simple bag-of-word approach, I will use the term-frequency
inverse-document-frequency (TF-IDF) matrix transformation which also
considers in how many documents a word appears out of all the documents
in the corpus. This is because words that appear too frequently across
all the documents in the corpus may not convey much meaning of each
doucment and thus should be treated as stop words. Due to the relatively
small size of the dataset (with only around 550 articles), I will also
adopt non-negative matrix factorisation (NMF) to reduce the
dimensionality of the dataset to prevent overfitting. According to
Stevens et al.
(\protect\hyperlink{ref-stevensExploringTopicCoherence2012}{2012})
(p.953), the matrix denoted as \emph{H} which captures the weight of
each topic (as columns) in each document (as rows) of the corpus can
help summarise the information of the articles in terms of which
topic(s) they primarily focus on. Thus, I will use NMF to transform the
TF-IDF matrix into another which summarises the news articles into a
certain number of pre-defined latent topics (the exact number will be
figured out in the \textbf{\emph{preprocessing}} section later), and
then use this matrix as part of the inputs for the models.

The majority of the implementation from data preprocessing, modelling to
validation will be done via the \texttt{scikit-learn} package in Python
(\protect\hyperlink{ref-pedregosaScikitlearnMachineLearning2011}{Pedregosa
et al. 2011}), whereas the tokenisation of the Chinese-language news
articles will be done by the \texttt{jieba} package which is designed
for natural language processing in Chinese
(\protect\hyperlink{ref-junyiJieba2021}{Junyi 2021}). For better
tokenisation of the articles, I will also remove stop words,
punctuations and hyper-links of the texts and add words related to Hong
Kong politics (downloaded from elgarteo
(\protect\hyperlink{ref-elgarteoHKPolDictHongKong2021}{2021})) and
asylum seekers into the dictionary. As for the gradient boosting model,
the \texttt{XGBoost} package
(\protect\hyperlink{ref-chenXgboostExtremeGradient2015}{Chen et al.
2015}) will be used since it is one of the most renowned machine
learning models in data science, and it also offers a lot of options for
tuning the model for better performance.

\hypertarget{operationalisation}{%
\section{Operationalisation}\label{operationalisation}}

\hypertarget{main-independent-variable}{%
\subsection{Main independent variable}\label{main-independent-variable}}

Starting from the main independent variable which classifies the
associated political camp of the media outlets, I will mainly refer to
the literature exploring how Hong Kong's media industry is intertwined
with politics and perform additional research when needed. I consulted
the articles by Fong
(\protect\hyperlink{ref-fongInbetweenLiberalAuthoritarianism2017}{2017})
and Lee (\protect\hyperlink{ref-leeChangingPoliticalEconomy2018}{2018})
on the development of post-1997 Hong Kong's media industry, of which
they have documented the co-optation of Hong Kong media outlets by the
Chinese Government (e.g.~the major shareholders holding positions in
such as Chinese People's Political Consultative Conference). Building on
the concept of ``allocative control'' (p.11) by Lee
(\protect\hyperlink{ref-leeChangingPoliticalEconomy2018}{2018}) about
media outlet owners having the ultimate decision on the personnel and
resources at the newsroom's disposal for reporting news which may lead
to self-censorship of the frontline journalists, I will code media
outlets whose owners are affiliated with the official Chinese and/or
Hong Kong Government organisations as belonging to the pro-Beijing camp.
As for those media owners with less apparent links with the Chinese or
Hong Kong Governments, I will decide the classification according to
whether the owners are known to be affiliated with a certain political
camp and how a given media outlet is perceived for its political stance.
Media outlets without apparent political stances are labelled as
``neutral''. \emph{Table 3.2} is the resulting classification scheme.
Note that due to the medium of publication of localist news media being
almost exclusively online, none of the outlets in that camp was included
in the dataset. Therefore, it would not be possible to examine how the
localist media outlets might report on asylum seekers and
non-refoulement claimants in news articles.

\begin{longtable}[]{@{}
  >{\raggedright\arraybackslash}p{(\columnwidth - 2\tabcolsep) * \real{0.4048}}
  >{\raggedright\arraybackslash}p{(\columnwidth - 2\tabcolsep) * \real{0.5952}}@{}}
\caption{Classification of the associated political camp of media
outlets in Hong Kong included into the sample, source: Fong
(\protect\hyperlink{ref-fongInbetweenLiberalAuthoritarianism2017}{2017});
Lee (\protect\hyperlink{ref-leeChangingPoliticalEconomy2018}{2018});
author's own research}\tabularnewline
\toprule
\begin{minipage}[b]{\linewidth}\raggedright
Associated political camp
\end{minipage} & \begin{minipage}[b]{\linewidth}\raggedright
Media outlets
\end{minipage} \\
\midrule
\endfirsthead
\toprule
\begin{minipage}[b]{\linewidth}\raggedright
Associated political camp
\end{minipage} & \begin{minipage}[b]{\linewidth}\raggedright
Media outlets
\end{minipage} \\
\midrule
\endhead
Pro-Beijing camp (directly owned by the Liaison Office or the HKSAR
Government) & Wen Wei Po, Hong Kong Government News, Hong Kong
Commercial Daily, Ta Kung Pao \\
Pro-Beijing Camp (owners affiliated with political institutions on
mainland China/ HKSAR Government) & Oriental Daily News, Sing Tao Daily,
Sing Pao, Headline Daily, Hong Kong Economic Journal, Hong Kong Economic
Times, am730, Sky Post, HK01 Newspaper \\
Pro-Democracy camp & Apple Daily, Kung Kao Po \\
Localist camp & N/A \\
Neutral & Metro Daily, Ming Pao Daily News \\
\bottomrule
\end{longtable}

\hypertarget{dependent-variable}{%
\subsection{Dependent variable}\label{dependent-variable}}

As mentioned before, the dependent variable of H\textsubscript{1} will
be the sentiment of the news articles towards asylum seekers in the
polarities of \emph{positive}, \emph{neutral} and \emph{negative}. But
how exactly should these three polarities mean in this context? Again
referring to the literature can help devise a sound coding scheme.
Drawing McKay, Thomas, and Warwick Blood
(\protect\hyperlink{ref-mckayAnyOneThese2011}{2011}), Cooper et al.
(\protect\hyperlink{ref-cooperMediaCoverageRefugees2017}{2017}) and I.
Ng, Choi, and Chan
(\protect\hyperlink{ref-ngFramingIssueAsylum2019}{2019}),
\emph{positive} articles usually acclaim the person or event in concern,
describe personal stories, report expert opinions without derogatory
terms or depict refugees favourably, \emph{neutral} articles simply
report an incident without an overt sentiment or opinion towards
particular stakeholders and with little interpretation by the
journalists, and \emph{negative} articles frame asylum seekers either as
bringing problems to the host society or use inaccurate terms to
delegitimise this group of population. I will follow these coding
criteria for the polarities of the news articles in the dataset since
these definitions suit the purpose of this study. The numerical
representation of the polarities in the dataset will be as follows,
namely, positive articles will be coded as \texttt{2}, neutral articles
will be coded as \texttt{1}, and negative articles will be coded as
\texttt{0}.

For the context of Hong Kong about the ``incorrect'' terms and
information used for describing asylum seekers, I will refer to expert
opinions about whether descriptions of asylum seekers in Hong Kong are
founded\footnote{A detailed discussion of the validity of the depiction
  of asylum seekers in Hong Kong can be found in I. Ng
  (\protect\hyperlink{ref-ngCriminalizingInnocentsSocial2020}{2020}) and
  I. Ng, Choi, and Chan
  (\protect\hyperlink{ref-ngFramingIssueAsylum2019}{2019}).}. For
example, I. Ng, Choi, and Chan
(\protect\hyperlink{ref-ngFramingIssueAsylum2019}{2019}) point out that
the term ``fake/bogus refugees'' (\emph{jia3 nan4min2)} should be
considered as both derogatory and inaccurate because those applying for
non-refoulement claims are not automatically qualified as refugees while
awaiting the final decisions of their applications. \emph{Therefore, the
polarity of news articles in which the news media directly refers
non-refoulement claimants as ``fake refugees'' in its reportage will be
coded as negative, but the report should be coded at least as neutral
instead if the news outlet is merely quoting someone who describes
non-refoulement claimants as ``fake refugees''.} Furthermore, I. Ng
(\protect\hyperlink{ref-ngCriminalizingInnocentsSocial2020}{2020})
mentions that non-refoulement claimants are forced to be illegal
immigrants by the policy design of the HKSAR Government so that this
group of population can be excluded from social rights that legal
residents enjoy. News articles equating non-refoulement claimants as
illegal immigrants will thus also be coded as negative since such
reportage is justifying the exclusion of non-refoulement claimants by
resorting to their illegality that had been forced upon by the HKSAR
Government.

\hypertarget{control-variables}{%
\subsection{Control variables}\label{control-variables}}

I will also incorporate some control variables to see if the association
between political camps and polarities of the news articles still
remains significant afterwards. For starters, the time when the articles
were published will be included as a control, though I will bin the
twelve months into four evenly split quarters (\texttt{Quarter}) with
the first quarter beginning in January and ending in March. Since 2019
was also the year when one of the most large-scale and long-lasting
pro-democracy protests broke out, It would be interesting to see how
asylum seekers may be portrayed during the peak of the protest.
Moreover, the presence of racial labels will be included as a control,
since it is possible that more overtly racialised discourses may be
associated with more negative sentiments towards asylum seekers in
general according to the literature. This variable will be named
\texttt{Racial\_label} in the dataset and will be binary, with
\texttt{1} meaning there is presence of racial labels in the main text
and \texttt{0} otherwise. Racial labels referring to both regional
(e.g.~South Asia, Africa) and national (e.g.~the nationalities included
in figure 2.2) will be counted. Lastly, the metadata of the news
articles, namely the character counts of titles and main texts, will
also be added to see if the lengths of the articles may be associated
with the sentiment towards asylum seekers.

\hypertarget{results}{%
\chapter{Results}\label{results}}

Placeholder

\hypertarget{exploratory-data-analysis-eda}{%
\section{Exploratory data analysis
(EDA)}\label{exploratory-data-analysis-eda}}

\hypertarget{how-does-the-number-of-news-articles-vary-by-political-camps-and-month}{%
\subsection{How does the number of news articles vary by political camps
and
month?}\label{how-does-the-number-of-news-articles-vary-by-political-camps-and-month}}

\hypertarget{polarities-of-the-news-articles}{%
\subsection{Polarities of the news
articles}\label{polarities-of-the-news-articles}}

\hypertarget{presence-of-racial-labels}{%
\subsection{Presence of racial labels}\label{presence-of-racial-labels}}

\hypertarget{character-lengths-of-news-articles-and-titles}{%
\subsection{Character lengths of news articles and
titles}\label{character-lengths-of-news-articles-and-titles}}

\hypertarget{sentiment-analysis}{%
\section{Sentiment analysis}\label{sentiment-analysis}}

\hypertarget{preprocessing}{%
\subsection{Preprocessing}\label{preprocessing}}

\hypertarget{training-the-model}{%
\subsection{Training the model}\label{training-the-model}}

\hypertarget{is-the-pro-beijing-camp-more-likely-to-portray-asylum-seekers-in-2019-more-negatively-than-other-outlets}{%
\section{Is the pro-Beijing camp more likely to portray asylum seekers
in 2019 more negatively than other
outlets?}\label{is-the-pro-beijing-camp-more-likely-to-portray-asylum-seekers-in-2019-more-negatively-than-other-outlets}}

\hypertarget{overall-magnitudes-of-shap-values-for-each-feature}{%
\subsection{Overall magnitudes of SHAP values for each
feature}\label{overall-magnitudes-of-shap-values-for-each-feature}}

\hypertarget{shap-values-of-predicting-negative-news-articles}{%
\subsection{SHAP values of predicting negative news
articles}\label{shap-values-of-predicting-negative-news-articles}}

\hypertarget{shap-values-of-predicting-neutral-news-articles}{%
\subsection{SHAP values of predicting neutral news
articles}\label{shap-values-of-predicting-neutral-news-articles}}

\hypertarget{shap-values-of-predicting-positive-news-articles}{%
\subsection{SHAP values of predicting positive news
articles}\label{shap-values-of-predicting-positive-news-articles}}

\hypertarget{discussion}{%
\section{Discussion}\label{discussion}}

\setcounter{chapter}{4}

\hypertarget{conclusion}{%
\chapter{Conclusion}\label{conclusion}}

In conclusion, it is found that at least in 2019, the political camp of
media outlets was associated with their attitudes towards asylum seekers
in Hong Kong. Specifically,

\begin{itemize}
\tightlist
\item
  I will sum up the findings after editing the main sections
\end{itemize}

\hypertarget{how-might-the-instigation-of-the-national-security-law-affect-the-public-discourse-on-asylum-seekers-in-hong-kong}{%
\section{How might the instigation of the National Security Law affect
the public discourse on asylum seekers in Hong
Kong?}\label{how-might-the-instigation-of-the-national-security-law-affect-the-public-discourse-on-asylum-seekers-in-hong-kong}}

Just a year after the anti-extradition law protest had started and once
again mobilised a huge section of Hong Kong's society against the
authority, the HKSAR Government promulgated the National Security Law in
July 2020 which aims to tighten the control over the political landscape
of Hong Kong. As a result, non-pro-Beijing forces have witnessed their
political influence being severely restricted, both in official
institutions and the society. Even though the Legislative Council had
been one of the few avenues where pro-democracy activists and
politicians could run for office (despite lacking substantive authority
to participate in policymaking by proposing bills), with the conclusion
of the recent 2021 Legislative Council election after an overhaul of the
electoral system which essentially permits candidacy only to the
``patriots'' loyal to the Beijing and HKSAR Governments
(\protect\hyperlink{ref-lauPatriotsOnlyHong2021}{Lau and Yam 2021}), the
door for pro-democracy camp to advocate alternative political discourses
has been shut. This means when pro-Beijing lawmakers propose non-binding
bills to combat the so-called ``fake refugee'' problems as Holden Chow
did back in 2016
(\protect\hyperlink{ref-chengLegislatureRejectsLawmaker2016}{Cheng
2016}), such bills will very unlikely be opposed by others, and the
voting results on the however unbinding bills may then be used by the
Government to justify more stringent measures against non-refoulement
claimants.

Even the presence of alternative media which strive to provide an
alternative narrative of news contrary to the state-coopted mainstream
media as Lee
(\protect\hyperlink{ref-leeChangingPoliticalEconomy2018}{2018})
describes may not be the case anymore. In June 2021, the flagship
pro-democracy newspaper \emph{Apple Daily} was forced to shut down after
the HKSAR Government had frozen the media outlet's asset under the
pretext of violating the National Security Law
(\protect\hyperlink{ref-tanAppleDailyHong2021}{Tan 2021}). Just days
before the year 2021 ended, the popular pro-democracy online media
\emph{Stand News} had to shut down as well after the Hong Kong police
raided its office and arrested senior staff
(\protect\hyperlink{ref-ngHongKongProdemocracy2021}{E. Ng and Pomfret
2021}), followed by the closure of another online pro-democracy news
outlets days after
(\protect\hyperlink{ref-HongKongProdemocracy2022}{{``Hong {Kong}:
{Pro-democracy} Website {Citizen News} Is Latest to Close''} 2022}). It
is likely that more alternative media may be pressured to defunct in the
future as well due to the threat from the National Security Law, and
this implies that alternative discourses will exist in the media
industry of Hong Kong. Consequently, since the pro-Beijing camp media in
general are more hostile to non-refoulement claimants, it is likely that
Hong Kong's media landscape will be dominated by more negative rhetoric
towards asylum seekers in the city given that pro-democracy outlets are
now finding increasing difficulties to operate without running into
legal repercussion.

With two major channels of formulating and propagating political
discourses being shut off, it can be expected that the pro-Beijing camp
will face much less counter-discourse from the oppositions while
attempting to frame certain issues, including on the status and
legitimacy of non-refoulement claimants. As the study by Boomgaarden and
Vliegenthart
(\protect\hyperlink{ref-boomgaardenHowNewsContent2009}{2009}) shows, the
tone of how asylum seekers are reported in news media is negatively
associated with the public attitude towards asylum seekers in later
periods, meaning that more pejorative coverage of refugees by the media
may cause the public to be more averse towards this group of vulnerable
population, and the effect of negative news coverage may become even
more salient when the levels of immigration and asylum seeking
applications are perceived to be high.

In short, the gradual decrease of

\hypertarget{appendix-the-echoes-of-the-code}{%
\chapter*{Appendix: The Echoes of the
Code}\label{appendix-the-echoes-of-the-code}}
\addcontentsline{toc}{chapter}{Appendix: The Echoes of the Code}

Placeholder

\markboth{References}{References}

\hypertarget{refs}{}
\begin{CSLReferences}{1}{0}
\leavevmode\vadjust pre{\hypertarget{ref-boomgaardenHowNewsContent2009}{}}%
Boomgaarden, Hajo G., and Rens Vliegenthart. 2009. {``How News Content
Influences Anti-Immigration Attitudes: {Germany},
1993â\texteuro ``2005.''} \emph{European Journal of Political Research}
48 (4): 516--42.

\leavevmode\vadjust pre{\hypertarget{ref-chenXgboostExtremeGradient2015}{}}%
Chen, Tianqi, Tong He, Michael Benesty, Vadim Khotilovich, Yuan Tang,
and Hyunsu Cho. 2015. {``Xgboost: Extreme Gradient Boosting.''} \emph{R
Package Version 0.4-2} 1 (4): 1--4.

\leavevmode\vadjust pre{\hypertarget{ref-chengLegislatureRejectsLawmaker2016}{}}%
Cheng, Kris. 2016. {``Legislature Rejects Lawmaker's Non-Binding Motion
to Combat 'Bogus' Refugees.''} \emph{Hong Kong Free Press HKFP}.
https://hongkongfp.com/2016/12/01/legislature-rejects-lawmakers-non-binding-motion-combat-bogus-refugees/.

\leavevmode\vadjust pre{\hypertarget{ref-cooperMediaCoverageRefugees2017}{}}%
Cooper, Samantha, Erin Olejniczak, Caroline Lenette, and Charlotte
Smedley. 2017. {``Media Coverage of Refugees and Asylum Seekers in
Regional {Australia}: {A} Critical Discourse Analysis.''} \emph{Media
International Australia} 162 (1): 78--89.

\leavevmode\vadjust pre{\hypertarget{ref-cuhkPublicEvaluationMedia2020}{}}%
CUHK. 2020. {``Public {Evaluation} on {Media Credibility}.''}
\emph{Centre for Communication and Public Opinion Survey}.
https://ccpos.com.cuhk.edu.hk/public-evaluation-on-media-credibility/.

\leavevmode\vadjust pre{\hypertarget{ref-elgarteoHKPolDictHongKong2021}{}}%
elgarteo. 2021. {``{HKPolDict}: {Hong Kong} Politics Dictionary for Text
Mining.''} \emph{GitHub}. https://github.com/elgarteo/HKPolDict.

\leavevmode\vadjust pre{\hypertarget{ref-erniWhoNeedsStrangers2012}{}}%
Erni, John Nguyet. 2012. {``Who Needs Strangers? {Un-imagining Hong Kong
Chineseness}.''} \emph{Chinese Journal of Communication} 5 (1): 78--87.

\leavevmode\vadjust pre{\hypertarget{ref-fongInbetweenLiberalAuthoritarianism2017}{}}%
Fong, Brian. 2017. {``In-Between Liberal Authoritarianism and Electoral
Authoritarianism: {Hong Kong}'s Democratization Under {Chinese}
Sovereignty, 1997\textendash 2016.''} \emph{Democratization} 24 (4):
724--50. \url{https://doi.org/10.1080/13510347.2016.1232249}.

\leavevmode\vadjust pre{\hypertarget{ref-heathBritishNationalIdentity2005}{}}%
Heath, Anthony F., and James R. Tilley. 2005. {``British National
Identity and Attitudes Towards Immigration.''} \emph{International
Journal on Multicultural Societies} 7 (2): 119--32.

\leavevmode\vadjust pre{\hypertarget{ref-hjermNationalIdentityComparison1998}{}}%
Hjerm, Mikael. 1998. {``National Identity: {A} Comparison of {Sweden},
{Germany} and {Australia}.''} \emph{Journal of Ethnic and Migration
Studies} 24 (3): 451--69.

\leavevmode\vadjust pre{\hypertarget{ref-HongKongProdemocracy2022}{}}%
{``Hong {Kong}: {Pro-democracy} Website {Citizen News} Is Latest to
Close.''} 2022. \emph{BBC News}, January.

\leavevmode\vadjust pre{\hypertarget{ref-junyiJieba2021}{}}%
Junyi, Sun. 2021. {``Jieba.''}

\leavevmode\vadjust pre{\hypertarget{ref-lauPatriotsOnlyHong2021}{}}%
Lau, Jessie, and Shui-yin Sharon Yam. 2021. {``{`{Patriots Only}'}:
{Hong Kong}'s {New Election System} in {Action}.''} \emph{The Diplomat}.
https://thediplomat.com/2021/11/patriots-only-hong-kongs-new-election-system-in-action/.

\leavevmode\vadjust pre{\hypertarget{ref-leeChangingPoliticalEconomy2018}{}}%
Lee, Francis LF. 2018. {``Changing Political Economy of the {Hong Kong}
Media.''} \emph{China Perspectives} 2018 (2018/3): 9--18.

\leavevmode\vadjust pre{\hypertarget{ref-linMakeChinaGreat2021}{}}%
Lin, Cong, and Liz Jackson. 2021. {``Make {China} Great Again: {The}
Blood-Based View of {Chineseness} in {Hong Kong}.''} \emph{Educational
Philosophy and Theory} 53 (9): 907--19.

\leavevmode\vadjust pre{\hypertarget{ref-lundbergSlundbergShap2022}{}}%
Lundberg, Scott. 2022. {``Slundberg/Shap.''}

\leavevmode\vadjust pre{\hypertarget{ref-mckayAnyOneThese2011}{}}%
McKay, Fiona H., Samantha L. Thomas, and R. Warwick Blood. 2011.
{``"{Any} One of These Boat People Could Be a Terrorist for All We
{Know}!" {Media} Representations and Public Perceptions of
â\texteuro\texttildelow{{Boat People\^a}}\texteuro\texttrademark{{Arrivals}}
in {Australia}.''} \emph{Journalism} 12 (5): 607--26.

\leavevmode\vadjust pre{\hypertarget{ref-ngHongKongProdemocracy2021}{}}%
Ng, Edmond, and James Pomfret. 2021. {``Hong {Kong} Pro-Democracy {Stand
News} Closes After Police Raids Condemned by {U}.{N}., {Germany}.''}
\emph{Reuters}, December.

\leavevmode\vadjust pre{\hypertarget{ref-ngCriminalizingInnocentsSocial2020}{}}%
Ng, Isabella. 2020. {``Criminalizing the Innocents: Social Exclusion of
Asylum-Seekers and Refugees in {Hong Kong}.''} \emph{Journal of Asian
Public Policy} 13 (3): 319--32.

\leavevmode\vadjust pre{\hypertarget{ref-ngFramingIssueAsylum2019}{}}%
Ng, Isabella, Sharice Fungyee Choi, and Alex Lihshing Chan. 2019.
{``Framing the {Issue} of {Asylum Seekers} and {Refugees} for {Tougher
Refugee Policyâ}\texteuro ''a {Study} of the
{Mediaâ}\texteuro\texttrademark s {Portrayal} in {Post-colonial Hong
Kong}.''} \emph{Journal of International Migration and Integration} 20
(2): 593--617. \url{https://doi.org/10.1007/s12134-018-0624-7}.

\leavevmode\vadjust pre{\hypertarget{ref-odohertyAsylumSeekersBoat2007}{}}%
O'Doherty, Kieran, and Amanda Lecouteur. 2007. {``{`{Asylum}
Seekers,'}{`boat People'} and {`Illegal Immigrants'}: {Social}
Categorisation in the Media.''} \emph{Australian Journal of Psychology}
59 (1): 1--12.

\leavevmode\vadjust pre{\hypertarget{ref-pedregosaScikitlearnMachineLearning2011}{}}%
Pedregosa, Fabian, Gaël Varoquaux, Alexandre Gramfort, Vincent Michel,
Bertrand Thirion, Olivier Grisel, Mathieu Blondel, Peter Prettenhofer,
Ron Weiss, and Vincent Dubourg. 2011. {``Scikit-Learn: {Machine}
Learning in {Python}.''} \emph{The Journal of Machine Learning Research}
12: 2825--30.

\leavevmode\vadjust pre{\hypertarget{ref-stevensExploringTopicCoherence2012}{}}%
Stevens, Keith, Philip Kegelmeyer, David Andrzejewski, and David
Buttler. 2012. {``Exploring Topic Coherence over Many Models and Many
Topics.''} In \emph{Proceedings of the 2012 Joint Conference on
Empirical Methods in Natural Language Processing and Computational
Natural Language Learning}, 952--61.

\leavevmode\vadjust pre{\hypertarget{ref-tanAppleDailyHong2021}{}}%
Tan, Yvette. 2021. {``Apple {Daily}: {The Hong Kong} Newspaper That
Pushed the Boundary.''} \emph{BBC News}, June.

\leavevmode\vadjust pre{\hypertarget{ref-vanatteveldtValiditySentimentAnalysis2021}{}}%
van Atteveldt, Wouter, Mariken A. C. G. van der Velden, and Mark Boukes.
2021. {``The {Validity} of {Sentiment Analysis}: {Comparing Manual
Annotation}, {Crowd-Coding}, {Dictionary Approaches}, and {Machine
Learning Algorithms}.''} \emph{Communication Methods and Measures} 15
(2): 121--40. \url{https://doi.org/10.1080/19312458.2020.1869198}.

\end{CSLReferences}


%%%%% REFERENCES

% JEM: Quote for the top of references (just like a chapter quote if you're using them).  Comment to skip.
% \begin{savequote}[8cm]
% The first kind of intellectual and artistic personality belongs to the hedgehogs, the second to the foxes \dots
%   \qauthor{--- Sir Isaiah Berlin \cite{berlin_hedgehog_2013}}
% \end{savequote}

% \setlength{\baselineskip}{0pt} % JEM: Single-space References
% 
% {\renewcommand*\MakeUppercase[1]{#1}%
% \printbibliography[heading=bibintoc,title={References}]}

\end{document}